\documentclass[a4paper,12pt]{article}
\usepackage[pdfborder={0 0 0}]{hyperref}
%%\usepackage[defaultsans]{droidsans}
%%\renewcommand*\familydefault{\sfdefault} %% Only if the base font of the document is to be typewriter style
\usepackage[T1]{fontenc}
\usepackage[francais]{babel}
\usepackage[utf8]{inputenc}
\usepackage{color}
\usepackage{lmodern}
\usepackage{amsmath}
\usepackage{amssymb}
\usepackage{mathrsfs}
% \usepackage{pdfpages}
\usepackage{graphicx}
\usepackage[top=2.5cm, bottom=3cm, left=2cm, right=2cm]{geometry}
\usepackage{variations}
\usepackage{colortbl}
\usepackage{textcomp}
\usepackage{eurosym}
\usepackage{tabularx}
\usepackage{lastpage}
\usepackage{fancyhdr}

\pagestyle{fancy}

\fancyfoot[L]{\textit{IPIWolf}}
\fancyfoot[R]{\textit{Documentation technique}}
\fancyfoot[C]{\textit{Page \thepage/\pageref{LastPage}}}


\begin{document}
\title{Documentation utilisateur d'IPIWolf}
\author{Vincent Stébé - Nicolas Adam}
\date{Juin 2015}

\maketitle

\newpage 
\newgeometry{top=2cm, bottom=2cm}

\tableofcontents
\vspace{1cm}
\restoregeometry
\newpage

\section{Présentation}
\subsection{Le logiciel \emph{IPIWolf}}
Le logiciel \emph{IPIWolf} permet de traiter des fichiers issus d'accéléromètres.
Il prend en entrée des fichiers (appelés \emph{Raw File}) du format suivant :
\begin{verbatim}
$File has 224955801 lines.                                                                                                                            
Logger date      Logger time     N       N prime         X       Y       Z 
15/05/2014      08:00:01        28      1       -0.141  -1.609  1.531
                                -0.109  -0.078  1.016
                                -0.109  -0.078  1.016
                                -0.094  -0.078  1
                                -0.094  -0.078  1.016
                                -0.094  -0.078  1
                                -0.094  -0.078  1.031
                                -0.094  -0.062  1.016
(***)
\end{verbatim}
Le logiciel lit alors ce fichier et permet de sélectionner une partie du signal en spécifiant une date de début et une date de fin.
Il permet ensuite de rééchantillonner ce signal avec une fréquence de votre choix, et de procéder à un filtrage de fréquence.

\subsection{Configuration système}
Le logiciel a été conçu sous GNU/Linux (distributions Ubuntu et ArchLinux).
Il est nécessaire d'installer les dépendances suivantes :
\begin{itemize}
 \item Qt SDK 5.4
 \item FFTW 3
\end{itemize}
Ces librairies fonctionnent sur la plupart des systèmes.
Pour compiler, il suffit ensuite soit d'utiliser \emph{QtCreator} soit d'ouvrir un terminal et d'executer les commandes suivantes :
\begin{verbatim}
 qmake
 make
\end{verbatim}
L'exécutable \emph{ipiwolf} est alors généré et prêt à être utilisé.
<<<<<<< HEAD



\section{}
=======
>>>>>>> a77f96e8198b2c90db56c7b7b3695499bf448cf4
\end{document}

\section{Utilisation}
\subsection{Rééchantillonnage}
